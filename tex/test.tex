\documentclass[letterpaper]{article}

% \errorcontextlines=5

\usepackage[
    fonts = {
        fa = Font Awesome 6 Pro Regular,
        fas = Font Awesome 6 Pro Solid,
        fab = Font Awesome 6 Brands
    },
    dualglyphFonts = {
        fad = Font Awesome 6 Duotone,
    }
]{glyph}

\setmainfont{Helvetica Neue}
\newfontfamily\FA{Font Awesome 6 Pro Regular}
% \newcommand\foobarIcon[1]{{\fafont\char#1}}

\begin{document}

\section{Variable Testing}
If there's variables that I've set that I want to test, they can go here.\par

\section{Command Testing}
Test different "internal" commands \(e.g., \verb|\fooGlyph{exclamation-mark}|\).
% Remember, \fooGlyph{} is a generated CS.  It can already have the font information
% embedded in it, because we have that information at the time of defining all of the
% control sequences.  So it could call something like 
% \__glyph_put_glyph{Font Name}{glyph-name}.  That function needs to know how to 
% place a glyph using the given engine (e.g., LuaLaTeX, XeTeX).
% Does the file glyph work? {\fafont\char"F15B}? Maybe yes, maybe no.
% \ExplSyntaxOn
% \__glyph_put_glyph:nnn{Foo}{single}{file}
% \ExplSyntaxOff
Is this a person walking? 
{
  \fontspec{Font Awesome 6 Pro Regular}
  \symbol{128694}
}

Who knows?\par

Next test of a person walking: 
\ExplSyntaxOn
\__glyph_put_glyph:nnn{Foo}{single}{person-walking}
% Debugging & testing
% \__glyph_debug_repeater:n{testingtesting123}
\ExplSyntaxOff

\section{Glyph Testing}
Test glyph commands here (e.g., \verb|\fooExclamationMark|).


\pagebreak
\section{List of Glyphs}
This is just a table of all of the glyphs font for each font loaded.

% \begin{multicols}{4}\noindent
%     \begin{luacode*}
%     local f = fontloader.open('/Users/cawaltrip/Library/Fonts/Font Awesome 6 Pro-Regular-400.otf')
%     local glyphs = {}
%     for i = 0, f.glyphmax - 1 do
%        local g = f.glyphs[i]
%        if g then
%            table.insert(glyphs, {name = g.name, unicode = g.unicode})
%        end
%     end
%     table.sort(glyphs, function (a,b) return (a.unicode < b.unicode) end)
%     for i = 1, #glyphs do
%        tex.sprint(glyphs[i].unicode .. ": ")
%        if (glyphs[i].unicode > 0) then
%            tex.sprint("{\\FA\\char" .. glyphs[i].unicode .. "}");
%        else
%         -- Here the updated code: the first glyph table in LuaTeX printing ALL glyphs in a font!
%         -- tex.sprint("{\\FA\\char\\directlua{tex.print(luaotfload.aux.slot_of_name(font.current(), [[" .. glyphs[i].name .. "]]))}}")
%         -- luaotfload.aux.slot_of_name(font.current(), [[" .. glyphs[i].name .. "]])
%          tex.sprint("OUTSIDE GLYPH RANGE")
%        end
%        tex.print(" {\\small(")
%        tex.print(-2, glyphs[i].name )
%        tex.sprint(')}\\\\')
%     end
%     fontloader.close(f)
    
%     \end{luacode*}
% \end{multicols}

\end{document}